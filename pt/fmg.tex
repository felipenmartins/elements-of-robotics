% !TeX root = er.tex

\chapter{Máquinas de Estados Finitas}\label{ch.fmg}

Os veículos Braitenberg e os algoritmos de seguimento de linha apresentados no Cap.~\ref{ch.reactive} demonstram comportamento reativo, em que a ação do robô depende dos valores \emph{correntes} retornados pelos sensores do robô, e não de eventos que aconteceram anteriormente. Na seção Seção.~\ref{s.sm} apresentamos o conceito de máquinas de estados finitas (em inglês, \emph{Finite State Machines}, FSM). As seções ~\ref{s.reactive-state}--\ref{s.search-and-approach} mostram como certos veículos Braitenberg podem ser implementados com FSMs. A seção~\ref{s.fsm-implementation} discute a implementação de FSMs usando variáveis de estado.

\section{Máquinas de Estados}\label{s.sm}

O conceito de estado é muito familiar. Considere uma torradeira: inicialmente, a torradeira está no estado \p{desligado}; quando você empurra a alavanca para baixo, uma transição é feita para o estado \p{ligado} e os elementos de aquecimento são ligados; finalmente, quando o timer expira, uma transição é feita de volta para o estado \p{desligado} e os elementos de aquecimento são desligados.

Uma \emph{máquina de estados finita} (FSM)\footnote{As máquinas de estados finitas também são chamadas de autômatos finitos.} consiste em um conjunto finito de \emph{estados} $s_i$ e um conjunto de \emph{transições} entre pares de estados $s_i, s_j$. Uma transição é etiquetada como \textit{condição}/\textit{ação}: uma condição que faz com que a transição seja feita, e uma ação seja executada quando a transição é feita.

As FSMs podem ser exibidas na forma de \emph{diagramas de estados}:
\begin{center}
\begin{tikzpicture}[node distance = 6mm and 5cm,align=left]
\node[draw,circle,minimum size=12mm] (off) {\p{off}};
\node[draw,circle,minimum size=12mm] (on) [right=of off] {\p{on}};
\draw[->,bend left=15] (off) to node[above] {\p{alavanca para baixo $\leadsto $ liga  aquecedor}} (on);
\draw[->,bend left=15] (on) to node[below] {\p{temporizador expira $\leadsto$ desliga aquecedor}} (off);
\draw[->] (-15mm,10mm) to (off);
\node[draw,below=of on,xshift=-30mm,yshift=-4mm] { \p{off = torradeira desligada} \\ \p{on = torradeira ligada}};
\end{tikzpicture}
\end{center}

Um estado é denotado por um círculo etiquetado com o nome do estado. Os estados recebem nomes curtos para economizar espaço e seus nomes completos são indicados em uma caixa ao lado do diagrama de estados. A seta de entrada denota o estado inicial. Uma transição é mostrada como uma seta do estado \emph{fonte} para o estado \emph{destino}. A seta é rotulada com a condição e a ação da transição. A ação não é continuada; por exemplo, a ação \p{giro à esquerda} significa ajustar os motores de modo que o robô gire para a esquerda, mas a transição para o próximo estado é feita sem esperar que o robô chegue a uma posição específica.

\section{Comportamento Reativo com Estados}\label{s.reactive-state}

A especificação a seguir descreve um veículo de Braitenberg cujo comportamento não é reativo:

\begin{quote}
\normalsize\noindent\textbf{Especificação (Persistente)}: O robô se move para a frente até detectar um objeto. Quando isso ocorre, ele se move para trás por um segundo e, em seguida, volta a mover-se para frente.
\end{quote}
A Figura~\ref{fig.persistent} mostra o diagrama de estados para esse comportamento.

\begin{figure}
\begin{center}
\begin{tikzpicture}[node distance = 2cm and 6cm,align=left]
\node[draw,circle,minimum size=12mm] (forwards)  {\textsf{\small frente}};
\node[draw,circle,minimum size=12mm] (backwards) [right=of forwards] {\textsf{\small trás}};
\draw[->,bend left=15] (forwards) to node[above] {\textsf{\small detectado $\leadsto$}\\\p{ajustar os motores para trás, ligar o timer}} (backwards);
\draw[->,bend left=15] (backwards) to node[below] {\textsf{\small timer expirou $\leadsto $ ajustar motores para frente}} (forwards);
\draw[->] (-15mm,15mm) to  (forwards);
\node at (0mm,18mm) {\textsf{\small verdadeiro $\leadsto $ ajustar motores para frente}};
\node[draw,above=of backwards,yshift=-1cm,xshift=-1cm] {\textsf{\small frente = robô move-se para a frente}\\\textsf{\small trás = robô move-se para trás}};
\end{tikzpicture}
\caption{Máquina de estados para o veículo de Braitenberg persistente}\label{fig.persistent}
\end{center}
\end{figure}

Inicialmente, quando o sistema é ligado, os motores são ajustados para avançar. (A condição \p{verdadeiro} é sempre verdadeira, portanto a ação é realizada incondicionalmente). No estado {frente}, se um objeto for detectado,
 a transição para o estado \p{trás} é realizada, o robô se move para trás e o timer é ajustado. Quando o tempo de um segundo passa, o timer expira e a transição para o estado de \p{frente} é realizada: o robô passa a se mover para frente. Se um objeto é detectado quando o robô está no estado \p{trás}, \emph{nenhuma} ação é executada pois não há transição relacionada a essa condição. Isso mostra que o comportamento não é reativo. Ao contrário, ele depende do evento que ocorre \emph{e} do estado atual do robô.

\begin{framed}
\act{Consistente}{consistent}
\begin{itemize}
\item Desenhe o diagrama de estados para o veículo de Braitenberg Consistente.
\begin{quote}
\normalsize\noindent\textbf{Especificação (Consistente):} O robô percorre quatro estados, mudando de estado uma vez a cada segundo: avançando, virando à esquerda, virando à direita, recuando.
\end{quote}
\end{itemize}
\end{framed}

\section{Busca e Abordagem}\label{s.search-and-approach}

Esta seção apresenta um exemplo mais complexo de um comportamento robótico que utiliza estados.

\begin{quote}
\normalsize\noindent\textbf{Especificação (Busca e abordagem):} O robô procura à esquerda e à direita ($\pm 45^{\circ}$). Quando ele detecta um objeto, aproxima-se dele e pára quando está perto do objeto.
\end{quote}

Há duas configurações dos sensores de um robô que permitem que ele procure à esquerda e à direita como mostrado nas Figs.~\ref{fig.separate-sensors},~\ref{fig.rotating-sensor}. Se os sensores forem fixados ao corpo do robô, o próprio robô tem que virar para a esquerda e para a direita; caso contrário, se o sensor for montado de modo que possa girar, o robô pode permanecer parado e o sensor é girado. Assumimos que o robô tem sensores fixos.

A Figura~\ref{fig.search-approach} mostra o diagrama de estados para o comportamento de busca e abordagem. O robô está inicialmente procurando à esquerda. Há dois novos conceitos que estão ilustrados no diagrama. Há um \emph{estado final} rotulado \p{fim} e denotado por um círculo duplo no diagrama. Uma máquina de estados finita é \emph{finita} no sentido de que contém um \emph{número finito de estados e transições}. Entretanto, seu \emph{comportamento} pode ser finito ou infinito. O FSM na Fig.~\ref{fig.search-approach} demonstra comportamento finito porque o robô irá parar quando encontrar um objeto e se aproximar dele. Este FSM também pode ter um comportamento infinito: se um objeto nunca for encontrado, o robô continua indefinidamente a procurar à esquerda e à direita.

O veículo de Braitenberg Persistente (Fig.~\ref{fig.persistent}) tem comportamento infinito porque ele continua a se mover sem parar. Uma torradeira também demonstra comportamento infinito porque você pode continuar torrando fatias de pão para sempre (até que você desligue a torradeira ou fique sem pão).

\begin{figure}
\begin{center}
\begin{tikzpicture}[node distance = 2.2cm and 5.5cm,align=left]
\node[draw,circle,minimum size=12mm] (left) {\p{esq}};
\node[draw,circle,minimum size=12mm] (right) [right=of left] {\p{dir}};
\node[draw,circle,minimum size=12mm] (detected) [below=of left] {\p{aprox}};
\node[draw,circle,minimum size=12mm,yshift=-1mm] (near1) [below=of right] {\p{fim}};
\node[draw,circle,minimum size=14mm] (near) [below=of right] {};
\draw[->,bend left=15] (left) to node[above] {\p{em $+45^{\circ$}} $\leadsto$ \p{girar à direita}} (right);
\draw[->,bend left=15] (right) to node[below] {\p{em $-45^{\circ$}} $\leadsto$ \p{girar à esquerda}} (left);
\draw[->] (left) to node[xshift=1mm,yshift=3mm] {\p{detectado} $\leadsto$ \p{ajustar motores para frente}} (detected);
\draw[->,bend left=15] (right) to node[xshift=5mm] {\p{detectado} $\leadsto$ \p{ajustar motores para frente}} (detected);
\draw[->] (detected) to node[below] {\p{objeto próximo} $\leadsto$ \p{acionar motores}} (near);
\draw[->] (-15mm,10mm) to (left);
\node at (-3mm,12mm) {\textsf{\small verdadeiro $\leadsto$ girar à esquerda}};
\node[draw] at (3.5,-5.5) { \p{esq = robô girando à esquerda para buscar}\\ \p{dir = robô girando à direita para buscar}\\\p{aprox = robô se aproximando do objeto}\\\p{fim = objeto encontrado}};
\end{tikzpicture}
\caption{Diagrama de estados para busca e abordagem}\label{fig.search-approach}
\end{center}
\end{figure}

O segundo novo conceito é o \emph{não determinismo}. Os estados \emph{esquerda} e \emph{direita} têm duas transições de saída, uma quando o robô gira até o limite do setor que está sendo pesquisado e outra ao detectar um objeto. O significado da indeterminação é que qualquer uma das transições de saída pode ser realizada. Há três possibilidades:
\begin{itemize}
\item O objeto é detectado e o robô ainda não girou até o limite do setor; neste caso, é feita a transição para o \p{aprox}.
\item O robô gira até o limite da do setor, mas um objeto não é detectado; neste caso, é feita a transição da esquerda para a direita ou da direita para a esquerda.
\item O robô atinge o limite de giro exatamente quando um objeto é detectado; neste caso, é feita uma transição arbitrária. Ou seja, o robô pode se aproximar do objeto ou pode mudar a direção da busca.
\end{itemize}
O comportamento não determinístico do terceiro caso pode fazer com que o robô não se aproxime do objeto quando este for detectado pela primeira vez, se este evento ocorrer ao mesmo tempo que o evento de atingir $\pm45^{\circ}$. Entretanto, após um curto período, as condições serão verificadas novamente e é provável que apenas um dos eventos ocorra.

\section{Implementação de Máquinas de Estados Finitas}\label{s.fsm-implementation}

Para implementar comportamentos com estados, variáveis devem ser utilizadas. O veículo Persistente (Sect.~\ref{s.reactive-state}) precisa de um timer para ativar um evento após um período de tempo ter expirado. Como explicado na seção Sect.~\ref{s.embedded}, um timer é uma variável que é definida para o período de tempo desejado. A variável é decrementada pelo sistema operacional e, quando chega a zero, ocorre um evento.

O algoritmo~\ref{alg.persistent} descreve como a FSM da Fig.~\ref{fig.persistent} é implementada. A variável \p{current} contém o estado corrente do robô; na conclusão do processamento de um gerenciador de eventos, o valor dessa variável é definido para o estado destino. Os valores de \p{current} são nomeados \p{fwd} e \p{back} para maior clareza, embora em um computador eles seriam representados por valores numéricos.

\begin{figure}
\begin{alg}{Persistente}{persistent}
&\idv{}integer timer&// Timer em millisegundos\\
&\idv{}states current \ass fwd& // Define estado: frente\\
\hline
\stl{}&left-motor-power \ass $100$&\\
\stl{}&right-motor-power \ass $100$&\\
\stl{}&loop&\\
\stl{}&\idc{}when current $=$ fwd and object detected in front&\\
\stl{}&\idc{}\idc{} left-motor-power \ass $-100$&\\
\stl{}&\idc{}\idc{} right-motor-power \ass $-100$&\\
\stl{}&\idc{}\idc{} timer \ass $1000$&\\
\stl{}&\idc{}\idc{} current \ass back&// Define estado: trás\\
\stl{}&&\\
\stl{}&\idc{}when current $=$ back and timer $=0$&\\
\stl{}&\idc{}\idc{} left-motor-power \ass $100$&\\
\stl{}&\idc{}\idc{} right-motor-power \ass $100$&\\
\stl{}&\idc{}\idc{} current \ass fwd&// Define estado: frente\\
\end{alg}
\end{figure}

\begin{framed}
\act{Persistente}{persistent}
\begin{itemize}
\item Implemente o comportamento Persistente.
\end{itemize}
\end{framed}

\begin{framed}
\act{Paranóico (direção alternada)}{paranoid2-state}
\begin{itemize}
\item Desenhe o diagrama de estados do veículo de Braitenberg Paranóico (direção alternada):
\begin{quote}
\normalsize\noindent\textbf{Especificação (Paranóico (direção alternada)):}
\begin{itemize}
\item Quando um objeto é detectado na frente do robô, o robô se move para a frente.
\item Quando um objeto é detectado à direita do robô, o robô gira à direita. 
\item Quando um objeto é detectado à esquerda do robô, o robô gira à esquerda. 
\item Se o robô estiver girando (mesmo que não detecte mais um objeto), ele alterna a direção de seu giro a cada segundo.
\item Quando nenhum objeto é detectado e o robô não está girando, o robô pára. 
\end{itemize}
\end{quote}
\item Implemente essa especificação. Além de uma variável que armazena o estado atual do robô, use outra variável com valores \p{esquerda} e \p{direita} para armazenar a direção em que o robô gira. Defina um temporizador com um período de um segundo. No gerenciador de eventos do temporizador, altere o valor da variável de direção para o valor oposto e reinicie o temporizador.
\end{itemize}
\end{framed}

O algoritmo~\ref{alg.search-and-approach} é um esboço da implementação do diagrama de estado na Fig.~\ref{fig.search-approach}.

\begin{figure}
\begin{alg}{Busca e abordagem}{search-and-approach}
&\idv{}states current \ass{} esq&\\
\hline
\stl{}&left-motor-power \ass $50$&// Girar à esquerda\\
\stl{}&right-motor-power \ass $150$&\\
\stl{}&loop&\\
\stl{}&\idc{}when object detected&// Quando objeto é detectado\\
\stl{}&\idc{}\idc{}if current $=$ esq&// Se estado atual é esq\\
\stl{}&\idc{}\idc{}\idc{}left-motor-power \ass $100$&// Vai para frente\\
\stl{}&\idc{}\idc{}\idc{}right-motor-power \ass $100$&\\
\stl{}&\idc{}\idc{}\idc{}current \ass{} aprox&// Atualiza estado\\
\stl{}&\idc{}\idc{}else if current $=$ dir&\\
\stl{}&\idc{}\idc{}\idc{}$\cdots$&\\
\stl{}&\idc{}when at $+45^{\circ}$&// Quando a $+45^{\circ}$\\
\stl{}&\idc{}\idc{}if current $=$ esq& Se estado atual é esq\\
\stl{}&\idc{}\idc{}\idc{}left-motor-power \ass $150$&// Vire à direita\\
\stl{}&\idc{}\idc{}\idc{}right-motor-power \ass $50$&\\
\stl{}&\idc{}\idc{}\idc{}current \ass{} dir&// Atualiza estado\\
\stl{}&\idc{}when at $-45^\circ$&// Quando a $-45^\circ$\\
\stl{}&\idc{}\idc{} $\cdots$&\\
\stl{}&\idc{}when object is very near&// Quando objeto está próximo\\
\stl{}&\idc{}\idc{}if current $=$ aprox&// Se estado atual é aprox\\
\stl{}&\idc{}\idc{}\idc{}$\cdots$&\\
%\stl{}&\idc{}\idc{}\idc{}left-motor-power \ass $0$&// Stop\\
%\stl{}&\idc{}\idc{}\idc{}right-motor-power \ass $0$&\\
%\stl{}&\idc{}\idc{}\idc{}current \ass{} found&\\
\end{alg}
\end{figure}

\begin{framed}
\act{Busca e abordagem}{search-and-approach}
\begin{itemize}
\item Preencha as linhas em falta no Algoritmo~\ref{alg.search-and-approach}.
\item Implemente o Algoritmo~\ref{alg.search-and-approach}.
\end{itemize}
\end{framed}

\section{Sumário}

A maioria dos algoritmos robóticos exige que o robô mantenha uma representação interna de seu estado atual. As condições que o robô usa para decidir quando mudar de estado e as ações tomadas ao mudar de um estado para outro são descritas pelas máquinas de estados finitas. As variáveis de estado são usadas para implementar máquinas de estados em programas.

\section{Leitura adicional}

O livro clássico sobre autômatos foi publicado em 1979 por John Hopcroft e Jeffrey D. Ullman; sua última edição é \cite{automata}. Para uma explicação detalhada sobre escolhas  \emph{arbitrárias} entre alternativas em não determinismo (Sect.~\ref{s.search-and-approach}), veja \cite[sect.~2.4]{pcdp2}.
